\documentclass[a4j,11pt,titlepage]{ltjsarticle}
\title{}
\date{\today}
\author{}
\usepackage{amsmath,amssymb} %言わずもがな
\usepackage{type1cm} %フォントサイズ自由変更
\usepackage{graphicx} % 画像
\usepackage[separate-uncertainty]{siunitx}
\usepackage[nobreak]{cite}

\usepackage[cache=false]{minted}
% \begin{minted}{LANG}
% 
% \end{minted}
% \setminted{frame=single,fontsize=\footnotesize,fontseries=courier,breaklines,linenos}
% \inputminted{LANG}{FILE}

\setminted{frame=single,fontsize=\small,fontseries=courier,breaklines}

\usepackage[version=4]{mhchem}
\usepackage{url}

% \usepackage[top=35mm,bottom=30mm,left=30mm,right=30mm]{geometry}  % Word標準設定
% \usepackage[top=25.4mm,bottom=25.4mm,left=19.05mm,right=19.05mm]{geometry}  % Wordやや狭い設定
% \usepackage[top=12.7mm,bottom=12.7mm,left=12.7mm,right=12.7mm]{geometry}  % Word狭い設定

\usepackage{physics}

\usepackage{enumerate} % \begin{enumerate}[{[1]}]で[1],[2]… \begin{enumerate}[(1)]でカッコつき数字


\everymath{\displaystyle}

\graphicspath{{./figure/}}

\begin{document}
\maketitle
% 参考文献は\citeで導入 bibtexを導入しているので情報はbibへ入力を
%%%%%%%%%%%%%%%%%%%%%%%%%%%%%%%%%%%%%%%%%%%%%%%%%%%%%%%%%
\section{目的}
\label{section:目的}


%%%%%%%%%%目的終わり%%%%%%%%%%

\section{原理}
\label{section:原理}


%%%%%%%%%%原理終わり%%%%%%%%%%

\section{方法}
\label{section:方法}


%%%%%%%%%%方法終わり%%%%%%%%%%

\section{実験結果}
\label{section:実験結果}


%%%%%%%%%%実験結果終わり%%%%%%%%%%

\section{考察}
\label{section:考察}


%%%%%%%%%%考察終わり%%%%%%%%%%

\section{感想}

\bibliography{sankou}
\bibliographystyle{junsrt}
\end{document}
